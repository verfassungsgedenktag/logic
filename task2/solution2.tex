\documentclass[12pt, a4paper]{article}

% кодировка и языки
\usepackage[T2A]{fontenc}
\usepackage[utf8]{inputenc}
\usepackage[english, russian]{babel}

% ams
\usepackage{amsthm}
\usepackage{amsfonts}
\usepackage{amsmath}
\usepackage{amssymb}

% окружения
\renewcommand{\qedsymbol}{\(\blacksquare\)}

\newtheoremstyle{task}%
    {}{}%
    {\slshape}{}%
    {\bfseries}{.}%
    { }{\thmname{#1}\thmnumber{ #2}\thmnote{ #3}}

\theoremstyle{task}
\newtheorem*{problem}{Задача}

% остальное
\frenchspacing
\binoppenalty=100000
\relpenalty=100000
\oddsidemargin=-0.5in
\textwidth=7.3in
\topmargin=-0.5in
\textheight=9.7in

% операторы
\DeclareMathOperator{\cov}{cov}
\DeclareMathOperator{\expected}{\mathbb{E}}
\DeclareMathOperator{\prob}{\mathbb{P}}
\DeclareMathOperator{\predicate}{P}

\usepackage[bb=boondox]{mathalfa}
\usepackage{eucal}
\usepackage{physics}
\usepackage{enumitem}
\usepackage{centernot}
\usepackage{algpseudocodex}

\newcommand{\cA}{\mathcal{A}}
\newcommand{\cB}{\mathcal{B}}
\newcommand{\cC}{\mathcal{C}}

\newcommand{\bA}{\mathbb{A}}
\newcommand{\bG}{\mathbb{G}}
\newcommand{\bN}{\mathbb{N}}
\newcommand{\bR}{\mathbb{R}}
\newcommand{\bQ}{\mathbb{Q}}
\newcommand{\bZ}{\mathbb{Z}}
\renewcommand{\emptyset}{\varnothing}


\begin{document}
    \section*{Домашнее задание №2, Марченко М.}

    \begin{problem}[1]
        Докажите, что вещественное число определимо в структуре \((\bR; =, +, \cdot, 0, 1)\) тогда и только тогда, когда оно алгебраическое. Охарактеризуйте вещественные числа, определимые в структуре \((\bR; =, +, \cdot, 0, 1)\).

        {\itshape Примите без доказательства, что в упорядоченном поле вещественных чисел любая формула равносильна подходящей бескванторной формуле}
    \end{problem}
    \begin{proof}[Решение]
        Термы в структуре \((\bR; =, +, \cdot, 0, 1)\) представляют собой многочлены с целыми коэффициентами. Из этого очевидным образом следует достаточность, обсудим необходимость.
        
        Если предикат \(x = \alpha\) задаётся атомарной формулой, она представляет собой \(t_1 = t_2\), где \(t_1, t_2\)~--- термы. Тогда \(\alpha\) является корнем \(t_1 - t_2\) --- многочлена с целыми коэффициентами.
        
        Если \(x = \alpha\) задаётся формулой вида \(\psi_1 \land \psi_2\), где \(\psi_1, \psi_2\) --- атомарные, перейдём к рассмотрению \(\psi_1, \psi_2\) и таким же образом заключим, что \(\alpha\) --- алгебраическое. Аналогично с \(\lor, \lnot\) и \(\to\).

        Если же \(x = \alpha\) задаётся формулой \(\forall y \enskip \psi\), где \(\psi\) --- бескванторная, можно зафиксировать значение \(y\) и получить, что \(\alpha\) является корнем некоторого многочлена с целыми коэффициентами. Аналогично с \(\exists\). \qed

        Из предыдущих рассуждений получаем, что для всех вещественных чисел, определимых в \((\bR; =, +, \cdot, 0, 1)\), соответствующий предикат задаётся формулой вида \(t = 0\), где \(t\) --- некоторый терм.
    \end{proof}

    \begin{problem}[2]
        Докажите, что комплексное число определимо в структуре \((\bC; =, +, \cdot, 0, 1)\) тогда и только тогда, когда оно рациональное.

        Примите без доказательства, что в поле комплексных чисел любая формула равносильна подходящей бескванторной формуле.
    \end{problem}
    \begin{proof}[Решение]
        TODO
    \end{proof}

    \begin{problem}[3]
        Докажите, что в стандартной модели арифметики \((\bN; =, +, \cdot)\) определимы: любое конкретное натуральное число; отношения строгого порядка и делимости; множество всех простых чисел; отношение быть простыми близнецами; множества степеней двойки, тройки, четвёрки, пятёрки.
    \end{problem}
    \begin{proof}[Решение]
        \begin{itemize}
            \item Число \(n\) определимо тогда и только тогда, когда определим предикат \(x = n\). Предикат \(x = 1\) задаётся формулой \(\forall n \enskip x \cdot n = n\). Индуктивно можно определить произвольное натуральное число: предикат \(x = n\) задаётся формулой \(x = (n - 1) + 1\);
            \item Предикат \(x < y\) задаётся формулой \(\exists n \enskip y = x + n\), а предикат \(x \mid y\) --- формулой \(\exists n \enskip y = x \cdot n\);
            \item Предикат <<\(x\) --- простое число>> задаётся формулой \((x \neq 1) \land (\forall n \enskip n \neq x \land n \neq 1 \to n \nmid x)\).
            \item Предикат <<\(x\) и \(y\) --- простые близнецы>> задаётся следующей формулой: \[
                (x \ \text{--- простое}) \land (y \ \text{--- простое}) \land (y = x + 2 \lor x = y + 2).
            \]
            \item Предикат <<\(x\) --- степень двойки>> задаётся следующей формулой: \[
                \forall n \enskip (n \mid x \land n \ \text{--- простое}) \to {n = 2}.
            \]
            Аналогично можно задать множество степеней тройки и пятёрки. Множество же степеней четвёрки можно задать индуктивно с помощью формулы \[
                (x = 1) \lor (\exists n \enskip x = 4 \cdot n \land n \ \text{--- степень четвёрки}).
            \]
        \end{itemize}
    \end{proof}

    \begin{problem}[4]
        Пусть \(A\) --- \(k\)-буквенный алфавит, где \(k \geqslant 2\). Определим бинарные отношения \(\leqslant_p, \leqslant_s, \leqslant_i\) и \(\preceq\) на \(A^*\) следующим образом:
        \begin{itemize}
            \item \(u \leqslant_p v\), если \(ux = v\) для некоторого \(x \in A^*\);
            \item \(u \leqslant_s v\), если \(xu = v\) для некоторого \(x \in A^*\);
            \item \(u \leqslant_i v\), если \(xuy = v\) для некоторых \(x, y \in A^*\);
            \item \(u \preceq v\), если \(u\) получается из \(v\) стиранием некоторых букв.
        \end{itemize}
        Докажите, что:
        \begin{enumerate}
            \item Отношение \(\leqslant_i\) определимо через отношения \(\leqslant_p\) и \(\leqslant_s\);
            \item Пустое слово определимо через любое из этих отношений;
            \item Множество всех слов фиксированной длины определимо через любое из этих отношений;
            \item Никакое фиксированное непустое слово не определимо через все эти отношения;
            \item Существует двухбуквенное слово, не определимое через отношения \(\leqslant_i, \preceq\) и однобуквенные слова;
            \item Опишите все слова, не определимые как в предыдущем вопросе.
        \end{enumerate}
    \end{problem}
    \begin{proof}[Решение]
        \begin{enumerate}
            \item Отношение \(\leqslant_i\) задаётся формулой \(\exists x \enskip u \leqslant_p x \land x \leqslant_s v\);
            \item Предикат \(x = \epsilon\) задаётся формулой \(\forall u \enskip x \leqslant u\), где \(\leqslant\) --- любое из введённых отношений;
            \item Соответствующий предикат задаётся формулой \(\forall x \ \forall y \enskip x \not\leqslant y\), где \(\leqslant\) --- любое из введённых отношений;
        \end{enumerate}
        TODO
    \end{proof}

    \begin{problem}[5]
        Докажите, что любой элемент структуры \((A^*; \leqslant_i)\), обогащённой константами для всех слов длины не более двух, определим. Охарактеризуйте группу автоморфизмов структуры \((A^*; \leqslant_i)\). Докажите аналогичные результаты для отношения \(\preceq\) вместо \(\leqslant_i\).   
    \end{problem}
    \begin{proof}[Решение]
        TODO
    \end{proof}
\end{document}