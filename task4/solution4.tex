\documentclass[12pt, a4paper]{article}

% кодировка и языки
\usepackage[T2A]{fontenc}
\usepackage[utf8]{inputenc}
\usepackage[english, russian]{babel}

% ams
\usepackage{amsthm}
\usepackage{amsfonts}
\usepackage{amsmath}
\usepackage{amssymb}

% окружения
\renewcommand{\qedsymbol}{\(\blacksquare\)}

\newtheoremstyle{task}%
    {}{}%
    {\slshape}{}%
    {\bfseries}{.}%
    { }{\thmname{#1}\thmnumber{ #2}\thmnote{ #3}}

\theoremstyle{task}
\newtheorem*{problem}{Задача}

% остальное
\frenchspacing
\binoppenalty=100000
\relpenalty=100000
\oddsidemargin=-0.5in
\textwidth=7.3in
\topmargin=-0.5in
\textheight=9.7in

% операторы
\DeclareMathOperator{\cov}{cov}
\DeclareMathOperator{\expected}{\mathbb{E}}
\DeclareMathOperator{\prob}{\mathbb{P}}
\DeclareMathOperator{\predicate}{P}

\usepackage[bb=boondox]{mathalfa}
\usepackage{eucal}
\usepackage{physics}
\usepackage{enumitem}
\usepackage{centernot}
\usepackage{algpseudocodex}

\newcommand{\cA}{\mathcal{A}}
\newcommand{\cB}{\mathcal{B}}
\newcommand{\cC}{\mathcal{C}}

\newcommand{\bA}{\mathbb{A}}
\newcommand{\bG}{\mathbb{G}}
\newcommand{\bN}{\mathbb{N}}
\newcommand{\bR}{\mathbb{R}}
\newcommand{\bQ}{\mathbb{Q}}
\newcommand{\bZ}{\mathbb{Z}}
\renewcommand{\emptyset}{\varnothing}


\begin{document}
\section*{Домашнее задание №4, Марченко М.}
    \begin{problem}[1]
        Пусть \(\bN\) --- стандартная структура натуральных чисел в сигнатуре \(\{=, +, \cdot\}\) и \linebreak \(\theory(\bN) \coloneq \{\varphi : \bN \models \varphi\}\). Нестандартной моделью арифметики называется модель теории \(\theory(\bN)\), не изоморфная \(\bN\). Докажите, что существует нестандартная модель арифметики и опишите структуру порядка в счётной нестандартной модели. 
    \end{problem}
    \begin{proof}[Решение]
        TODO
    \end{proof}

    \begin{problem}[2]
        Будут ли (конечно) аксиоматизируемыми следующие классы структур (подходящей сигнатуры):
        \begin{itemize}
            \item всех групп; всех конечных групп; всех бесконечных групп; всех абелевых групп; всех циклических групп; всех групп без кручения;
            \item всех полей; всех конечных полей; всех полей фиксированной характеристики; всех бесконечных полей; всех алгебраически замкнутых полей; всех алгебраически замкнутых полей фиксированной характеристики;
            \item всех упорядоченных полей; всех конечных упорядоченных полей; всех упорядоченных полей фиксированной характеристики; всех вещественно замкнутых упорядоченных полей?
        \end{itemize}
    \end{problem}
    \begin{proof}[Решение]
        \begin{itemize}
            \item Обозначим через \(\cG\) теорию, состоящую из аксиом группы, то есть из предложений \[
                \forall x \ \forall y \ \forall z \ (x \cdot y) \cdot z = x \cdot (y \cdot z); \quad \forall x \ (x + e = x) \land (e + x = x); \quad \forall x \ x \cdot x^{-1} = e. 
            \]
            Тогда класс всех групп (обозначим через \(\cK\)) в точности равен  \(\modclass(\cG)\), а значит, конечно аксиоматизируем. \qed

            Пользуясь теоремой компактности, можно доказать, что если теория имеет модель сколь угодно большой конечной мощности, то она имеет и бесконечную модель. Из этого утверждения следует, что класс \(\cK_{<\infty}\) конечных групп не аксиоматизируем, так как для любого натурального \(n\) существует группа из \(n\) элементов. \qed

            Рассмотрим предложения \(\varphi_n = \exists x_1 \ldots \exists x_n \ (x_1 \neq x_2) \land \ldots (x_{n-1} \neq x_n)\) (здесь перебираются все пары элементов \(x_1, \ldots, x_n\)). Пусть \(\cG_{\infty} = \cG \cup \{\varphi_n : n \in \bN\}\), тогда класс \(\cK_{\infty}\) всех бесконечных групп в точности равен \(\modclass(\cG_{\infty})\), а значит, аксиоматизируем. При этом он не конечно аксиоматизируем, так как тогда \(\cK_{<\infty} = (\structures_\sigma \setminus \cK_{\infty}) \cap \cK\) был бы аксиоматизируем. \qed

            Пусть \(\cG_A = \cG \cup \{\forall x \ \forall y \ x \cdot y = y \cdot x\}\), тогда класс \(\cK_A\) всех абелевых групп есть в точности \(\modclass(\cG_A)\). \qed

            Пусть \(\cK_c\) --- класс всех циклических групп. Покажем, что у \(\theory(\cK_c)\) есть модель, не являющаяся циклической группой (откуда \(\cK_c\) --- не аксиоматизируем). Будем использовать сокращение \(x^n = x \cdot \ldots \cdot x\) \(n\) раз. Возьмём сигнатуру \(\sigma' = \sigma \cup \{c\}\), где \(c\) --- новый константный символ, и рассмотрим \(\sigma'\)-теорию \[
                \cT = \theory(\cK_c) \cup \{\forall x \ x^n \neq c : n \in \bN\}.
            \]
            Покажем, что у любого конечного подмножества \(T \subset \cT\) есть модель. Положим \[
                N = \max \{n \in N : (\forall x \ x^n \neq c) \in T\} + 1,
            \]
            тогда достаточно взять \(\sigma'\)-обогащение циклической группы порядка \(N\), где \(c\) интерпретируется как \(e\). По теореме компактности у всего \(\cT\) есть некоторая модель \(\bA\). Тогда \(\sigma\)-обеднение этой модели будет моделью для \(\theory(\cK_c)\), но не будет циклической группой, так как \(c^\bA\) не представляется в качестве степени некоторого \(x\). \qed

            Пусть \(\cK_\lnot\) --- класс всех групп без кручения. Рассмотрим предложения \(\varphi_n = \forall x \ x \neq e \to x^n \neq e\) и теорию \(T = \cG \cup \{\varphi_n : n \in \bN\}\). Тогда \(\cK_\lnot = \modclass(T)\), то есть аксиоматизируем. Покажем, что при этом он не конечно аксиоматизируем.

            TODO \qed

            \item Пусть теория \(\cF\) состоит из аксиом поля (их всего \(9\)). Тогда класс \(\cK\) всех полей в точности равен \(\modclass(\cF)\), а потому конечно аксиоматизируем. \qed
            
            TODO \qed

            Рассмотрим предложения \(\varphi_n = (1 + 1 + \ldots + 1 = 0)\) (\(n\) единиц). Тогда класс полей нулевой характеристики есть в точности \(\modclass(T_0)\), где \(T_0 = \{\lnot \varphi_n : n \in \bN\}\). Зафиксируем любую ненулевую характеристику \(m\), тогда класс полей характеристики \(m\) есть в точности \(\modclass(T_m)\), где \[
                T_m = \{\lnot \varphi_n : n < m\} \cup \{\varphi_m\}.
            \] 
            Выходит, класс полей любой ненулевой характеристики конечно аксиоматизируем. TODO (про нулевую) \qed

            TODO \qed

            TODO
        \end{itemize}
    \end{proof}

    \begin{problem}[3]
        Докажите, что:
        \begin{itemize}
            \item любой аксиоматизируемый класс структур замкнут относительно элементарной эквивалентности и ультрапроизведений;
            \item если предложение логически следует из данного множества предложений, то оно логически следует из некоторого конечного подмножества данного множества;
            \item класс структур данной сигнатуры конечно аксиоматизируем тогда и только тогда, когда он сам и его дополнение (в классе всех структур данной сигнатуры) аксиоматизируемы.
        \end{itemize}
    \end{problem}
    \begin{proof}[Решение]
        \begin{itemize}
            \item Пусть \(\bA \in K = \modclass(T)\) (то есть \(\bA \models T\)) и \(\bA \equiv \bB\). Так как элементарно эквивалентные структуры удовлетворяют одни и те же предложения, то \(\bB \in K\).
            
            Пусть \(\bA_i \in K\), где \(i \in I\), и пусть \(F\) --- ультрафильтр на \(I\). По теореме об ультрафильтре \[
            \bA_F \models \varphi \iff \{i : \bA_i \models \varphi\} \in F.
            \]
            Так как \(\bA \models T\) для любого \(i\), то \(\{i : \bA_i \models T\} = I \in F\), откуда \(\bA_F \models T\). \qed
            
            \item Если предложение \(\varphi\) логически следует из некоторой теории \(T\), то \(\modclass(T \cup \{\lnot \varphi\}) = \emptyset\). Тогда по теореме компактности существует конечное \(T_n \subseteq T\) такое, что \(\modclass(T_n \cup \{\lnot \varphi\}) = \emptyset\), что значит, что \(\varphi\) логически следует из теории \(T_n\).
            \item Необходимость. Пусть \(K = \modclass(\varphi)\), где \(\varphi\) --- предложение, тогда \(\overline{K} = \structures_\sigma \setminus K = \modclass(\lnot \varphi)\).
            
            Достаточность. Пусть \(K = \modclass(T)\), \(\overline{K}= \modclass(T')\) для некоторых теорий \(T\) и \(T'\). Тогда \[
                \modclass(T \cup T') = \modclass(T) \cap \modclass(T') = \emptyset,
            \]
            откуда по теореме компактности существуют конечные \(T_n \subseteq T\) и \(T'_n \subseteq T'\) такие, что \[
                \modclass(T_n \cup T'_n) = \modclass(T_n) \cap \modclass(T'_n) = \emptyset.
            \]
            Тогда справедливо следующее: \[
                \modclass(T) \subseteq \modclass(T_n) \subseteq \overline{\modclass(T'_n)} \subseteq \overline{\modclass(T')} \subseteq \modclass(T),
            \]
            откуда \(\modclass(T) = \modclass(T_n)\).
        \end{itemize}
    \end{proof}
\end{document}