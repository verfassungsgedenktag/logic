\documentclass[12pt, a4paper]{article}

% кодировка и языки
\usepackage[T2A]{fontenc}
\usepackage[utf8]{inputenc}
\usepackage[english, russian]{babel}

% ams
\usepackage{amsthm}
\usepackage{amsfonts}
\usepackage{amsmath}
\usepackage{amssymb}

% окружения
\renewcommand{\qedsymbol}{\(\blacksquare\)}

\newtheoremstyle{task}%
    {}{}%
    {\slshape}{}%
    {\bfseries}{.}%
    { }{\thmname{#1}\thmnumber{ #2}\thmnote{ #3}}

\theoremstyle{task}
\newtheorem*{problem}{Задача}

% остальное
\frenchspacing
\binoppenalty=100000
\relpenalty=100000
\oddsidemargin=-0.5in
\textwidth=7.3in
\topmargin=-0.5in
\textheight=9.7in

% операторы
\DeclareMathOperator{\cov}{cov}
\DeclareMathOperator{\expected}{\mathbb{E}}
\DeclareMathOperator{\prob}{\mathbb{P}}
\DeclareMathOperator{\predicate}{P}

\usepackage[bb=boondox]{mathalfa}
\usepackage{eucal}
\usepackage{physics}
\usepackage{enumitem}
\usepackage{centernot}
\usepackage{algpseudocodex}

\newcommand{\cA}{\mathcal{A}}
\newcommand{\cB}{\mathcal{B}}
\newcommand{\cC}{\mathcal{C}}

\newcommand{\bA}{\mathbb{A}}
\newcommand{\bG}{\mathbb{G}}
\newcommand{\bN}{\mathbb{N}}
\newcommand{\bR}{\mathbb{R}}
\newcommand{\bQ}{\mathbb{Q}}
\newcommand{\bZ}{\mathbb{Z}}
\renewcommand{\emptyset}{\varnothing}


\begin{document}
    \section*{Домашнее задание №1, Марченко М.}

    \begin{problem}[1]
        Запишите аксиомы теории групп предложениями следующих сигнатур: \(\{=, \cdot \}\), \(\{=, \cdot, e \}\), \(\{=, \cdot, e, ^{-1} \}\), \(\{=, P \}\), где \(P\) --- трёхместный предикатный символ, обозначающий график умножения. Какая их этих сигнатур лучше соответствует изложению теории групп? Сделайте синтаксический анализ полученных формул.
    \end{problem}
    \begin{proof}[Решение]
        \begin{itemize}
            \item \(\{=, \cdot \}\):
            \begin{enumerate}
                \item \(\forall a \ \forall b \ \forall c \enskip (a \cdot b) \cdot c = a \cdot (b \cdot c)\),
                \item \(\exists e \ \forall a \enskip e \cdot a = a \land a \cdot e = a\),
                \item \(\forall a_1 \ \exists a_2 \ \forall b \enskip (a_1 \cdot a_2) \cdot b = b \land (a_2 \cdot a_1) \cdot b = b \land b \cdot (a_1 \cdot a_2) = b \land b \cdot (a_2 \cdot a_1) = b\).
            \end{enumerate}
            \item \(\{=, \cdot, e\}\):
            \begin{enumerate}
                \item \(\forall a \ \forall b \ \forall c \enskip (a \cdot b) \cdot c = a \cdot (b \cdot c)\),
                \item \(\forall a \enskip e \cdot a = a \land a \cdot e = a\),
                \item \(\forall a_1 \ \exists a_2 \enskip a_1 \cdot a_2 = e \land a_2 \cdot a_1 = e\).
            \end{enumerate}
            \item \(\{=, \cdot, e, ^{-1}\}\):
            \begin{enumerate}
                \item \(\forall a \ \forall b \ \forall c \enskip (a \cdot b) \cdot c = a \cdot (b \cdot c)\),
                \item \(\forall a \enskip e \cdot a = a \land a \cdot e = a\),
                \item \(\forall a_1 \ \exists a_2 \enskip a_2 = a_1^{-1}\).
            \end{enumerate}
            \item \(\{=, P\}\) (пусть \(\bG\) --- \(\sigma\)-структура, где \(\sigma = \{=, P\}\)):
            \begin{enumerate}
                \item \(\forall a \ \forall b \ \forall c \enskip \bG \models \predicate(a \cdot b, c, a \cdot (b \cdot c))\),
                \item \(\exists e \ \forall a \enskip \bG \models \predicate(e, a, a) \land \bG \models \predicate(a, e, a)\),
                \item \(\forall a_1 \ \exists a_2 \ \forall b \enskip \bG \models \predicate(a_1 \cdot a_2, b, b) \land \bG \models \predicate(a_2 \cdot a_1, b, b) \land \bG \models \predicate(b, a_1 \cdot a_2, b) \land \bG \models \predicate(b, a_2 \cdot a_1, b)\).
            \end{enumerate}
        \end{itemize}
        \qed

        Для изложения теории групп лучше всего подходит сигнатура \(\{=, \cdot, e, ^{-1}\}\), так как содержит ключевые для неё понятия нейтрального и обратного элементов. \qed

        Синтаксический анализ аксиом теории групп, записанных в сигнатуре \(\{=, \cdot, e, ^{-1}\}\):
        \begin{enumerate}
            \item Термы: \(a\), \(b\), \(c\), \(a \cdot b\), \((a \cdot b) \cdot c\), \(b \cdot c\), \(a \cdot (b \cdot c)\).
            
            Подформулы: \((a \cdot b) \cdot c = a \cdot (b \cdot c)\), \(\forall c \enskip (a \cdot b) \cdot c = a \cdot (b \cdot c)\), \(\forall b \ \forall c \enskip (a \cdot b) \cdot c = a \cdot (b \cdot c)\), \(\forall a \ \forall b \ \forall c \enskip (a \cdot b) \cdot c = a \cdot (b \cdot c)\).

            \item Термы: \(a\), \(e\), \(e \cdot a\), \(a \cdot e\).

            Подформулы: \(e \cdot a = a\), \(a \cdot e = a\), \(e \cdot a = a \land a \cdot e = a\), \(\forall a \enskip e \cdot a = a \land a \cdot e = a\).

            \item Термы: \(a_1\), \(a_2\), \(a_1^{-1}\).
            
            Подформулы: \(a_2 = a_1^{-1}\), \(\exists a_2 \enskip a_2 = a_1^{-1}\), \(\forall a_1 \ \exists a_2 \enskip a_2 = a_1^{-1}\).
        \end{enumerate}
    \end{proof}

    \begin{problem}[2]
        Докажите основные равносильности. Докажите, что любую формулу можно привести к предварённому виду, в котором бескванторная часть является ДНФ бескванторных формул.
    \end{problem}
    \begin{proof}[Решение]
        Основные равносильности легко доказать, построив таблицы истинности для формул слева и справа от \(\equiv\) и убедившись, что они одинаковые. \qed

        Используя равносильности для кванторов, последние можно <<протащить>> в начало формулы, после чего с помощью первых десяти равносильностей привести бескванторную часть формулы к ДНФ. 
    \end{proof}

    \begin{problem}[3]
        Существует ли алгоритм, вычисляющий значение \(\varphi^\bA\) по любой конечной сигнатуре \(\sigma\), \(\sigma\)-предложению \(\varphi\) и конечной \(\sigma\)-структуре \(\bA\). Если да, оцените время работы вашего алгоритма.
    \end{problem}
    \begin{proof}[Решение]
        Ниже приведён псевдокод алгоритма:
        \begin{algorithmic}[1]
            \Function{evaluate}{\(\varphi\), \(\bA\)}
                \If{\(\varphi = \predicate(t_1, \ldots, t_n)\)}
                \Comment{если \(\varphi\) --- атомарная, вычислить значения термов и \(\varphi\)}
                    \State \Return \(\predicate(t_1, \ldots, t_n)\) 
                \EndIf
                \item[]
                
                \LComment{если \(\varphi\) содержит логические связки, её значение вычисляется рекурсивно}
                \If{\(\varphi = \lnot \psi\)}
                    \State \Return \(\lnot\)\Call{evaluate}{\(\psi\), \(\bA\)}
                \EndIf
                \If{\(\varphi = \psi_1 \land \psi_2\)}
                    \State \Return \Call{evaluate}{\(\psi_1\), \(\bA\)} \(\land\) \Call{evaluate}{\(\psi_2\), \(\bA\)}
                \EndIf
                \If{\(\varphi = \psi_1 \lor \psi_2\)}
                    \State \Return \Call{evaluate}{\(\psi_1\), \(\bA\)} \(\lor\) \Call{evaluate}{\(\psi_2\), \(\bA\)}
                \EndIf
                \If{\(\varphi = \psi_1 \to \psi_2\)}
                    \State \Return \(\lnot\) \Call{evaluate}{\(\psi_1\), \(\bA\)} \(\lor\) \Call{evaluate}{\(\psi_2\), \(\bA\)}
                \EndIf
                \item[]

                \LComment{если \(\varphi\) содержит кванторы, перебираем все \(|\sigma|\) возможных значений переменных}
                \If{\(\varphi = \forall x \ \psi\)}
                    \ForAll{\(a \in \bA\)}
                        \If{\(\lnot\)\Call{evaluate}{\(\psi(a)\), \(\bA\)}}
                            \State \Return Л
                        \EndIf
                    \EndFor
                    \State \Return И
                \EndIf
                \If{\(\varphi = \exists x \ \psi\)}
                    \ForAll{\(a \in \bA\)}
                        \If{\Call{evaluate}{\(\psi(a)\), \(\bA\)}}
                            \State \Return И
                        \EndIf
                    \EndFor
                    \State \Return Л
                \EndIf
            \EndFunction
        \end{algorithmic}

        Если \(n\) --- количество переменных в \(\varphi\), а \(|\varphi|\) --- длина формулы \(\varphi\), то время работы алгоритма можно оценить как \(O(|\varphi| \cdot |\sigma|^n)\).
    \end{proof}

    \begin{problem}[4]
        Докажите, что отношение \(\leqslant\) не определимо в структуре \((\bZ; =, +)\), а операция \(+\) не определима в структуре \((\bZ; =, \leqslant)\).
    \end{problem}
    \begin{proof}[Решение]
        Пусть \(g \colon x \mapsto -x\), тогда \(g\) --- автоморфизм. Так как \(a \leqslant b \centernot\implies -a \leqslant -b\), то отношение~\(\leqslant\) не инвариантно, а следовательно, не определимо. \qed

        Функция определима тогда и только тогда, когда определим её график. Но с помощью \(=\) и \(\leqslant\) мы можем определять только вертикальные и горизонтальные прямые, а также участки плоскости, ограниченные этими прямыми, в то время, как график \(+\) --- наклонная прямая.
    \end{proof}

    \begin{problem}[5]
        Напишите предложения сигнатуры \(\{=, \leqslant\}\) такие, что любая модель получившегося множества предложений является плотным линейным порядком с наименьшим, но без наибольшего элемента. Докажите, что любые две счётные модели этого множества изоморфны. Верно ли это для моделей мощности континуум?
    \end{problem}
    \begin{proof}[Решение]
        Просто составим требуемое множество из определения плотного линейного порядка с наименьшим, но без наибольшего элемента:
        \begin{itemize}
            \item \(\forall x \ \forall y \enskip x \neq y \to (x \leqslant y \lor y \leqslant x)\),
            \item \(\forall x \ \forall y \enskip x \leqslant y \to (\exists z \enskip x \leqslant z \leqslant y)\),
            \item \(\exists m \ \forall x \enskip m \leqslant x\),
            \item \(\forall x \ \exists y \enskip x \leqslant y\). \qed
        \end{itemize}
        
        Рекурсивно построим изоморфизм \(\varphi\) между произвольной счётной моделью этого множества и структурой \((\bQ_+; =, \leqslant)\). Пусть \(\varphi(m) = 0\); если мы нашли образы для элементов \(\{a_i\}_{i=0}^k\) (пусть этот список отсортирован по возрастанию), то образ элемента \(a_{k+1}\) найдём следующим образом: если \(a_k \leqslant a_{k+1}\), то выберем произвольное рациональное число, большее \(\varphi(a_k)\) в качестве образа \(a_{k+1}\). В противном случае найдём в списке рассмотренных такие числа, что \(a_i \leqslant a_{k+1} \leqslant a_{i+1}\) и выберем произвольное рациональное число между \(\varphi(a_i)\) и \(\varphi(a_{i+1})\) в качестве образа \(a_{k+1}\). \qed

        TODO
    \end{proof}
\end{document}